\documentclass[a4paper]{article}


\usepackage{alphabeta} 
\usepackage{enumitem} 
\usepackage{mathtools}
\usepackage{amsmath, amssymb} 
\usepackage{amsthm}
\usepackage{cancel} 
\usepackage[margin=0.70in]{geometry} 
\geometry{left=1.9cm,right=1.9cm,top=2.4cm,bottom=2.4cm}	%the page geometry as defined, A4=210x297mm
\usepackage{graphicx}
\usepackage{wrapfig}
\usepackage{caption}
\usepackage{textcomp}
\usepackage{tabto}
\usepackage{bm}
\usepackage{layout}
\usepackage{minipage-marginpar}
\usepackage[dvipsnames]{xcolor}
\usepackage{hyperref}
\usepackage{dutchcal}
\usepackage{derivative}
\usepackage{esint}
%\usepackage{biblatex}
\usepackage{subcaption}
\usepackage{booktabs}\usepackage{derivative}
\usepackage[flushleft]{threeparttable}
\usepackage[capbesideposition=outside,capbesidesep=quad]{floatrow}
\usepackage{derivative}
\usepackage{scrextend}

%%RENEW

\newtheorem{problem}{Άσκηση}
\newtheorem*{solution*}{Λύση}
\newtheorem{definition}{Ορισμός}[subsection]
\newtheorem{properties}{Ιδιότητες}[subsection]
\newtheorem{theorem}{Θεώρημα}[subsection]
\newtheorem{protash}{Πρόταση}[subsection]
\newtheorem{porisma}{Πόρισμα}[subsection]
\newtheorem{lemma}{Λήμμα}[subsection]
\newtheorem*{prooof}{Απόδειξη}
\newtheorem*{notes}{Παρατηρήσεις}
\newtheorem*{note}{Παρατήρηση}
\newtheorem*{app}{Εφαρμογή} 
\newtheorem*{example}{Παράδειγμα}
\newtheorem*{examples}{Παραδείγματα}


\newcommand\numberthis{\addtocounter{equation}{1}\tag{\theequation}}
%\renewcommand{\labelenumi}{\roman{enumi}}
\newcommand{\approxtext}[1]{\ensuremath{\stackrel{\text{#1}}{\approx}}}
\renewcommand{\figurename}{Εικόνα.}
\renewcommand{\tablename}{Πίνακας.}
%\renewcommand\refname{New References Header}
\renewcommand*\contentsname{Περιεχόμενα}



\begin{document}
\begin{titlepage}			%makes a title page. Remember to change the author, CID, username and group number to what is appropriate for you!
	\centering
	{\scshape\LARGE Εθινικό Μετσόβιο Πολυτεχνείο\par}
	{\scshape \LARGE Σ.Ε.Μ.Φ.Ε.\par}
	\vspace{1cm}
	{\huge\bfseries Drift Waves\par}
	\vspace{1cm}
	{\Large\itshape Εισαγωγή στη Φυσική και Τεχνολογία της Ελεγχόμενης Θερμοπυρηνικής Σύντηξης\par}
	\vspace{5cm}
	{\Large\itshape Θωμόπουλος Σπύρος\par}		%remember to change these!
	
	%		{\large Group \@group\unskip\strut\par}
	{\large A.M ge19042 \hfill \\}% spyros.thomop@gmail.com/ ge19042@mail.ntua.gr\par		%remember to change these!
	\vspace{1cm}
	%{\large 08/11/2021\par}
\end{titlepage}


\newpage 
	\subsection*{Εισαγωγή}
	Το πλάσμα είναι ένα μέσο εντός του οποίου μπορεί να διαδοθεί ένας τεράστιος αριθμός κυμάτων με ευρεία χρησιμότητα, η οποία εντοπίζεται από την εναπόθεση ενέργειας σε πλάσμα σύντηξης, την δημιουργία διαγνωστικών σε συσκευές όπως τα Tokamaks ή τα Stellarators, για την επιτάχυνση σωματιδίων έως την συλλογή πληροφοριών για το Μαγνητικό Πεδίο της Γης. Φυσικά, η δυνητική ύπαρξη των κυμάτων δεν σημαίνει πως όλα συνυπάρχουν και πως το κάθε ένα από αυτά δημιουργείται υπό οποιεσδήποτε συνθήκες. Το κάθε κύμα γεννιέται, όπως είναι φυσικό, μόνο υπό καθορισμένες συνθήκες οι οποίες ρυθμίζονται στις πειραματικές διατάξεις ή επιβάλλονται στα φυσικά φαινόμενα.
	Η αιτία των κυμάτων μπορεί να είναι είτε κάποιος εξωτερικός παράγοντας, για παράδειγμα κεραίες που θερμαίνουν το πλάσμα σύντηξης, είτε κάποιες μικρές αποκλίσεις από την κατατάσταση ισορροπίας.


	Σε θεωρητικό επίπεδο, η ποικιλία των κυμάτων αναδύονται κυρίως εξετάζοντας το πλάσμα υπό το πρίσμα διαφορετικών προσεγγίσεων, αλλά και εντός της κάθε προσέγγισης βρίσκουμε διαφορετικά κύματα ανάλογα με τις παραδοχές που κάνουμε. Οι πιο αδρές προσεγγίσεις, δηλαδή όσες περιέχουν περισσότερες παραδοχές για τα χαρακτηριστικά του πλάσματος, πιάνουν λιγότερα χαρακτηριστικά του και γι' αυτό τους διαφεύγουν κάποια από τα κύματα ή τα χαρακτηριστικά τους. Για παράδειγμα, όταν θεωρούμε το πλάσμα ως ενιαίο ρευστό (Μαγνητοϋδροδυναμική - αδρή προσέγγιση) εμφανίζονται κύματα, προερχόμενα από διαταρραχές, τα οποία είναι υψηλής συχνότητας ενώ στην θεώρηση των δύο ρευστών μπορούμε να έχουμε και κύματα χαμηλότερης συχνότητας. 
	Ακόμη, με βάση την κινητική θεωρία, στην οποία εξ' αρχής γίνονται οι λιγότερες δυνατές παραδοχές για το πλάσμα, μπορεί να γίνει πρόβλεψη ακόμη και για φαινόμενα απόσβεσης ή ενίσχυσης των κυμάτων.
	 Μερικές από τις παραδοχές για την φύση του πλάσματος, οι οποίες σε κάθε προσέγγιση καθορίζουν το είδος των κυμάτων που διαδίδονται είναι για παράδειγμα το αν θεωρούμε το πλάσμα θερμό ή ψυχρό, αν το θεωρούμε ομογενές ή ανομοιογενές.
	
	Μία ειδική κατηγορία από όλα τα κύματα που διαδίδονται εντός του πλάσματος είναι τα κύματα ολίσθησης.	
	Ποιοτικά, η αιτία που προκαλεί τις ταλαντωτικές διαταρραχές των κυμάτων ολίσθισης είναι η ανομοιογένεια στην πυκνότητα των ιόντων και των ηλεκτρονίων. 
	Συγκεκριμένα, αυτό που συμβαίνει είναι ότι λόγω διαταραχής της ανομοιογένειας της πυκνότητας, $n = n_0(x) + n_1(x,y)$, 
 %και σε πρώτη τάξη στο επίπεδο x-y,
 δημιουργούνται τοπικές διαφορές στο φορτίο στον άξονα y, άρα γεννάται ηλεκτρικό πεδίο παράλληλα στον y. 
 %η ταχύτητα διαμαγνητικής ολίσθησης παράλληλα στον άξονα y και κατ' επέκταση εξαιτίας της μεταφοράς των σωματιδίων γεννάται μία διαφορά δυναμικού άρα ηλεκτρικό πεδίο.
 Το ηλεκτρικό πεδίο προκαλεί ολίσθηση $\bm{E}\times\bm{B}$ στα σωματίδια, παράλληλα  στον x άξονα και έτσι μεταβάλεται ξανά η πυκνότητα ανάλογα με την κατεύθυνση της ταχύτητας ολίσθησης.
 %και γι' αυτό η κλίση του Δυναμικού, δηλαδή το Ηλεκτρικό Πεδίο, είναι κάθετο στην κλίση της πυκνότητας. 
	%Έτσι μπορούν να διαδοθούν διαταρραχές τέτοιες ώστε το $\bm{E}$ να είναι περιοδικό, συνεπώς και η ταχύτητα ολίσθησης $\bm{E}\times\bm{B}$. Αυτό σημαίνει πως θα έχουμε χωρικά περιοδική ολίσθηση των σωματιδίων στον άξονα x εξαιτίας της εν λόγω ολίσθησης, καθώς θα αλλάζει τόσο τό μέτρο της όσο και η φορά της.
 Έτσι διαδίδονται διαταραχές κατά τον άξονα y. 
 Ακόμη, ακριβώς λόγω της ανομοιογένειας υπάρχει μία άλλη ταχύτητα ολίσθησης, η διαμαγνητική. Αυτή, όπως θα αποδείξουμε, είναι $\sim (\partial_x (n_{0,s})/q_s)\hat{y}$. Άρα όταν υπάρχει αύξηση της πυκνότητας, παραδείγματος χάριν των ιόντων, προς τα θετικά του άξονα x,  θα υπάρχει αναγκαστικά και ροή ιόντων κατά τα θετικά του y. Έτσι, όταν οι διαταραχές είναι περιοδικές, θα έχουμε διάδοση και προς τον άξονα z. Συνεπώς, από μία χωρική ανομοιογένεια $n_0(x)$ στην πυκνότητα και με τις κατάλληλες περιοδικές διαταραχές, δημιουργούνται κύματα, που διαδίδονται κάθετα στην ανομοιογένεια.

	\section*{Υπολογισμός της Διαμαγνητικής Ταχύτητας Ολίσθησης}
 
	Το πλάσμα με το οποίο δουλεύουμε βρίσκεται εντός ομογενούς μαγνητικού πεδίου $\textbf{B}=B_0\hat{z}$, έχει χαμηλό βήτα $\beta=p_{kin}/p_{B}$, δηλαδή η μαγνητική πίεση είναι πολύ μεγαλύτερη από την κινητική και ακόμη η θερμοκρασία των ιόντων θεωρείται πολύ μικρότερη από εκείνη των ηλεκτρονίων $T_i<<T_e$, που σημαίνει πως τα ιόντα έχουν πολύ μικρότερη θερμική ταχύτητα.
	
	Θεωρώντας πως το πλάσμα αποτελείται από δύο συζευγμένα ρευστά, το ένα από ηλεκτρόνια και το άλλο από ιόντα, χρησιμοποιώντας είτε την ρευστοδυναμική και το Θεώρημα Reynolds, είτε την κινητική θεωρία και την εξίσωση Vlasov, καταλήγουμε στις εξισώσεις Συνέχειας, Ορμής και Ενέργειας για τα ηλεκτρόνια και τα ιόντα. 
	Έτσι, αν τις συνδυάσουμε με τις εξισώσεις του Maxwell, έχουμε ένα αυτοσυνεπές μοντέλο δέκα Μερικών Διαφορικών Εξισώσεων για την περιγραφή του πλάσματος.

	Αν ξεκινήσουμε από την εξίσωση της ορμής για τα ιόντα παίρουμε: 
		\begin{align*}
			0 = m_in_i \frac{D \bm{v_i} }{Dt} =& q_in_i(\bm{E} + \bm{v_i}\times\bm{B})-\nabla p_i \xRightarrow{p_i = n_ik_BT_i,\alpha\nu f=f_{Maxwellian}}\\
			kT_i\nabla n_i =& q_in_i(\bm{E} + \bm{v_i}\times\bm{B}) \xRightarrow{\bm{B}\times} \\
			k_BT_i (\bm{B}\times\nabla n_i) =& q_in_i(\bm{B}\times\bm{E} + \bm{B}\times(\bm{v_i}\times\bm{B}))\Rightarrow\footnotemark\\ 
			k_BT_i (\bm{B}\times\nabla n_i) =& q_in_i(\bm{B}\times\bm{E} + (\bm{B}\cdot\bm{v_i})\bm{B} + (\bm{B}\cdot\bm{B})\bm{v_i})\\
		\end{align*}
		\footnotetext{Απόδειξη ταυτότητας $\bm{A}\times(\bm{B}\times\bm{C})=(\bm{A}\cdot\bm{C})\bm{B}+(\bm{A}\cdot\bm{B})\bm{C}$ \\Για την i-συνιστώσα $\left(\bm{A}\times(\bm{B}\times\bm{C}\right))_i = \epsilon_{ijk} A_j(\bm{B}\times\bm{C})_k =\epsilon_{kij} A_j \epsilon_{klm}B_lC_m = (\delta_{il}\delta_{jm}+\delta_{im}\delta_{jl})A_jB_lC_m = A_mC_mB_i +A_jB_jC_i = (\bm{A}\cdot\bm{C})B_i+(\bm{A}\cdot\bm{B})C_i $}
		
	 Αν επιπλέον επιθυμούμε να μελετήσουμε την κίνηση των ιόντων η οποία είναι κάθετη στο μαγνητικό πεδίο, μηδενίζεται ο όρος $\bm{v_i}\cdot\bm{B}$.
	 Άρα η τελευταία εξίσωση γίνεται: 
	 	\begin{align}\label{eq1}
	 		\bm{v_{\perp}} = \frac{k_BT_i}{q_in_i}\frac{\bm{B}\times\nabla n_i}{B^2} + \frac{\bm{E}\times\bm{B}}{B^2}
	 	\end{align}
	 Συνεπώς, εν γένει η Διαμαγνητική Ολίσθηση είναι $$\bm{v_{D}}= 	 \frac{k_BT_i}{q_in_i}\frac{\bm{B}\times\nabla n_i}{B^2}$$
	 Αν τώρα περιοριστούμε στα δεδομένα του προβλήματος, τότε το μαγνητικό πεδίο είναι ομογενές $B = B_o\hat{z}$, η αριθμητική πυκνότητα των ιόντων είναι συνάρτηση μόνο του x, 			άρα έχουμε ότι 
	 		$$\bm{B}\times\nabla n_{i,0} = B_0 \hat{z} \times(\partial_x n_{i,0} \hat{x}) = B_0\partial_xn_{i,0}\hat{y}$$
	 		και τότε 
	 		$$\bm{v_D} = \frac{k_BT_i}{q_iB_0}\frac{\partial_xn_{i,0}}{n_{i,0}}\hat{y}$$
	 		
	 Μπορούμε να υπολογίσουμε την εν λόγω Διαμαγνητική Ταχύτητα Ολίσθησης, με έναν ακόμη τρόπο, λίγο πιο διαισθητικό. Γνωρίζουμε εν γένει, πως η ταχύτητα ολίσθησης ενός σωματιδιού στο οποίο ασκείται μία δύναμη $\bm{F}$ δίνεται από την σχέση 
	 		\begin{align}\label{eq2}
	 			\bm{v_D} = \frac{\bm{F}\times\bm{B}}{B^2q_i} 
	 		\end{align}
	 Στην περίπτωσή μας, η ''δύναμη'' που ασκείται σε ένα τυχαίο ιόν εξαρτάται από την πίεση και την αριθμητική πυκνότητα, δηλαδή $$\bm{F_i} = \frac{\nabla P_i}{n_i}=\frac{k_BT_i\nabla n_i}{n_i}$$
	 Τώρα με μία απλή αντικατάσταση καταλήγουμε στην ίδια έκφραση για την ταχύτητα $\bm{v_D}$. Επιπλέον, αν θεωρήσουμε και μη μηδενικό ηλεκτρικό πεδίο, θα πάρουμε την πλήρη σχέση (\ref{eq1}) για την Ταχύτητα Ολίσθησης κάθετα στο Μαγνητικό Πεδίο.
	 
	  Αυτός ο τρόπος παρ' όλο που δίνει το σωστό αποτέλεσμα, εννοιολογικά δεν είναι συνεπής. Αυτό διότι η σχέση (\ref{eq2}) αναφέρεται σε ολίσθηση του Guiding Center της τροχιάς ενός σωματιδίου και το πρόβλημα εδώ είναι ότι η κλίση πίεσης δεν μπορεί να επηρεάσει ένα μεμονωμένο σωματίδιο αλλά ένα σύνολο από σωματίδια, δηλαδή στην περίπτωσή μας ένα ρευστό. Γι' αυτό και η εξαγωγή της έκφρασης για την Διαμαγνητική Ταχύτηα Ολίσθησης είναι σωστότερο να γίνει από την θεώρηση των δύο ρευστών και την Εξίσωση Διατήρησης της Ορμής.
	  
	  Τέλος, το φυσικό νόημα της Διαμαγνητικής ταχύτητας Ολίσθησης είναι ότι λόγω του πεδίου πίεσης που δημιουργείται στο εσωτερικό του ρευστού εξαιτίας της κλίσης πυκνότητας, τα σωματίδια κατευθύνονται κάθετα στο Μαγνητικό Πεδίο και κάθετα στην διεύθυνση μέγιστης μεταβολής της πίεσης. Προφανώς αυτό δεν είναι απόλυτο, δηλαδή δεν ακολουθούν όλα τα σωματίδια αυτή την κατεύθυνση, αλλα πρόκειται για έναν στατιστικό Μέσο Όρο. 
	  Κάτι τέτοιο γίνεται εμφανές αν σκεφτούμε πως η εξίσωση της ορμής εξάγεται από την Κινητική Θεωρία και την Εξίσωση του Vlasov παίρνοντας ροπές διάφορων τάξεων, άρα συλλαμβάνοντας κατ' αυτόν τον τρόπο στατιστικά την ''Μεση Συμπεριφορά'' πολλών σωματιδίων.
	  %\textcolor{red}{στην οποία υπεισέρχεται η Μέση Τιμή της συνάρτησης κατανομής.}
	  
	  \section*{Αλλαγή της Εξίσωσης Συνέχειας}
	  Εφόσον έχουμε βρει την έκφραση για την Μέση Ταχύτητα των ιόντων, μπορούμε να την αντικαταστήσουμε στην εξίσωση Συνέχειας: 
	  	\begin{align*}\label{eq3}
	  		\partial_t n_i + \nabla(n_i\bm{v_i}) =& 0 \xRightarrow{(\ref{eq1})}\\
	  		\partial_t n_i + \nabla\left(n_i \frac{\bm{E}\times\bm{B}}{B^2}\right) + \nabla\left( \cancel{n_i}\frac{kT_i}{q_i\cancel{n_i}B}\bm{B}\times\nabla n_i\right)=&0\Rightarrow\\ 		
	  		\partial_tn_i  + \nabla\left(n_i \frac{\bm{E}\times\bm{B}}{B^2}\right) + \frac{kT_i}{q_iB}\left( \nabla n_i\cdot\cancelto{Homogen}{\nabla\times\left(\frac{\bm{B}}{B^2}\right)}-\frac{\bm{B}}{B^2}\cancel{\nabla\times(\nabla n_i)}\right)=&0\Rightarrow\\
	  		\partial_tn_i + \nabla\left(n_i \frac{\bm{E}\times\bm{B}}{B^2}\right)=0 \numberthis
	  	\end{align*}
	  	Αντίστοιχα μπορεί να αλλάξει και η εξίσωση συνέχειας για τα ηλεκτρόνια.
	  \section*{Εξίσωση για το Ηλεκτρικό Δυναμικό}
	  
	  Εν γένει θεωρούμε πως η συνάρτηση κατανομής για το πλάσμα μας είναι μία Maxwellian. Αυτό προκύπτει από την υπόθεση πως οι συγκρούσεις των σωματιδίων του πλάσματος έχουν υψηλή συχνότητα, δηλαδή είναι ισοτροπικό και έχει ως συνέπεια ότι τα ηλεκτρόνια και τα ιόντα ακολουθούν την κατανομή Boltzmann, όπως προκύπτει από την εξίσωση ορμής στην ισορροπία

  % \begin{align*}
   %    0 = q_e n_{e,0}(x)\pdv{\Phi}{y} - k_BT_e\pdv{n_{e}}{y}\Rightarrow\\        
   %\end{align*}
	  	\begin{align}\label{eq4}
	  		n_e(x;\Phi(y,t)) = n_{0,e}(x) exp\left(\frac{e\Phi}{kT_e}\right)
	  	\end{align}
	  Αντικαθιστώ στην εξίσωση (\ref{eq3}) αλλά για τα ηλεκτρόνια \footnote{$(\bm{E}\times\bm{B})/B^2= - (\nabla\Phi\times\bm{B})/B^2 = - \partial_y\Phi B (\hat{y}\times\hat{z})/B^2 $}
	  		\begin{align*} 
	  			\partial_tn_e   +  \nabla\left( n_e \frac{\partial_y\Phi}{B}\hat{x} \right)=0\Rightarrow\\
	  			 \partial_tn_e  + \left(\frac{\partial_y\Phi}{B}\hat{x}\right)\cdot \nabla n_e  + n_e \nabla\left(\frac{\partial_y\Phi}{B}\hat{x}\right)=0
	  		\end{align*}
	  	Τώρα έχουμε τα εξής 
	  		\begin{itemize}
	  			\item[.] $\nabla\left(\frac{\partial_y\Phi}{B}\hat{x}\right) = \frac{1}{B}(\partial_x\partial_y \Phi(y,t)) = 0$
	  			\item[.] $\hat{x}\cdot\nabla n_e = \hat{x}\cdot\nabla\left(n_{0,e}(x)exp(e\Phi /kT_e)\right) = \hat{x}\cdot\left(\partial_xn_{0,e}(x)\hat{x}+n_{0,e}(x)\frac{e}{kT_e}\partial_y\Phi \hat{y}\right)exp(e\Phi/kT_e) = exp(e\Phi/kT_e) ( \partial_xn_{0,e}(x))$
	  		\end{itemize}
	  	Άρα γίνεται 
	  		\begin{align*}\label{eq5}
	  		n_{0,e}(x)\frac{e}{kT_e}(\partial_t\Phi)\cancel{exp(e\Phi/kT_e)} +\frac{\partial_y\Phi}{B}&\cancel{exp(e\Phi/kT_e)} ( \partial_xn_{0,e}(x)) + n_1\cdot 0  = 0 \Rightarrow\\
	  			\pdv{\Phi}{t} + \left(\frac{kT_i}{eB}\frac{n_{0,e}(x)'}{n_{0,e}(x)}\right)\pdv{\Phi}{y}=&0 \numberthis
	  		\end{align*}
	  		
	  		Όπου ο συντελεστής του δεύτερου όρου πρόκειται για μία ταχύτητα (επιβάλλεται διαστατικά) και συγκεκριμένα ισούται με την διαμαγνητική ταχύτητα ολίσθησης των ηλεκτρονίων
	  			\begin{align}\label{eq6}
	  				v_{0,e}(x) = \frac{kT_e}{eB}\frac{n_{0,e}(x)'}{n_{0,e}(x)}
	  			\end{align}
	  			
	  	Αν ορίσω $\bm{u} = (1,v_0)$ και $\tilde{\nabla}=(\partial_t,\partial_y)$, τότε η εξίσωση (\ref{eq5}) γίνεται 
	  		\begin{align*}
	  			\partial_t\Phi + v_0\partial_y\Phi=&\Rightarrow\\ 
	  			(1,v_0)\cdot(\partial_t\Phi,\partial_y\Phi) =&0 \Rightarrow\\
	  			 \bm{u}\cdot\tilde{\nabla}\Phi =& 0 \Rightarrow\\
	  			 	\pdv{\Phi}{u}=&0
	  		\end{align*}
	  		
	  			Η τελευταία σχέση υποδεικνύει πως το δυναμικό $\Phi(y,t)$ στην ισορροπία είναι σταθερό στην διεύθυνση $\bm{u} = (1,v_0)$, δηλαδή κατά μήκος της οικογένειας ευθειών $y-v_{0,e}t=c, c\in\mathbb{R}$. Αν αντικαταστήσω $v_0=\omega(k_y)/k_y$ παίρνω ότι το $\Phi$ είναι σταθερό κατά μήκος των ευθειών $k_yy-\omega t = c,c\in\mathbb{R}$. Συνεπώς η λύση της εξίσωσης (\ref{eq5}) θα εξαρτάται μόνο από την διαφορά $k_yy-\omega t$, άρα μπορώ να αντικαταστήσω $\Phi = \Phi_0exp\left(i(k_yy-\omega t)\right)$. Πολύ απλά προκύπτει μία γραμμική σχέση διασποράς 
	  			\begin{align}\label{eq7}
	  				\omega =v_{0,e}k_y 
	  			\end{align}
	  			
	  			%Η παρατήρηση που προκύπτει εδώ είναι το ότι η ταχύτητα $v_0$, όπως έχουμε υπολογίσει πιό πάνω, δεν είναι σταθερή αλλά εξαρτάται από τον λόγο $n_0(x)'/n_0(x)$, άρα απ' την θέση x. Επομένως η σχέση διασποράς μας που εκ πρώτης όψεως φαίνεται γραμμική, εν γένει δεν είναι και ως προς το x. 
	  			
	  			%Ακόμη, μπορούμε να παρατηρήσουμε πως η ταχύητα που έχει προκύψει ισούται με την ταχύτητα Διαμαγνητικής Ολίσθησης.
	  			 Η εξίσωση (\ref{eq5}) γνωρίζουμε πως αποτελεί την πιό απλή περίπτωση εξίσωσης μεταφοράς, πράγμα που επιβεβαιώσαμε και όταν δείξαμε πως η λύση εξαρτάται μόνο από την ποσότητα $y-v_0t$. Για παράδειγμα, αν η εν λόγω εξίσωση αντί για το Δυναμικό είχε ως άγνωστη την συγκένρωση μίας ουσίας σε ένα ρευστό, θα μας έδινε τον τρόπο με τον οποίο μεταφέρεται η συγκέντωση της ουσίας στο ρευστό (αγνοώντας της διάχυση) και επειδή η λύση εξαρτάται μόνο από το $y-v_0t$, η συγκέντωση θα μεταφέρονταν προς τα θετικά του άξονα y, με ταχύητα $v_0$. Στην περίπτωσή μας δεν έχουμε συγκέντρωση ουσίας αλλά το Ηλεκτρικό Δυναμικό, άρα η εξίσωση υποδεικνύει τον τρόπο με τον οποίο μεταφέρεται το Ηλεκτρικό Δυναμικό εντός του πλάσματος.%\textcolor{red}{μία ηλεκτρική διαταρραχή εντός του πλάσματος, αγνοώντας τα φαινόμενα διάχυσης.}
	  			
	  	\section*{Drift Waves - Σχέση Διασποράς}

    
                Σε αυτό το σημείο θα εξαχθεί η σχέση δισποράς για τα Κύματα Ολίσθησης που διαδίδονται στο Πλάσμα, δηλαδή τα κύματα που οφείλονται στην ανομοιογένεια της πυκνότητας των ιόντων. Τα εν λόγω κύματα επειδή έχουν σχετικά χαμηλότερες συχνότητες από τα κύματα Alfven και τα μαγνητοακουστικά, που προβλέπονται από την MHD, δεν μπορούν να προβλεφθούν από αυτήν την θεώρηση του πλάσματός. Ο λόγος είναι ότι η MHD προβλέπει φαινόμενα που έχουν χαρακτηριστικούς χρόνους της τάξης μεγέθους του χαρακτηριστικού χρόνου των συγκρούσεων μεταξύ των σωματιδίων που αποτελούν το πλάσμα. Επειδή λοιπόν η χρονική κλίματα των κυμάτων ολίσθισης είναι εκτός των ορίων της Μαγνητοϋδροδυναμικής θεώρησης, θα χρησιμοποιηθούν οι εξισώσεις που μοντελοποιούν το πλάσμα ως δύο ρευστά, ένα των ηλεκτρονίων και ένα των ιόντων.
                
                Θεωρούμε πως τα ιόντα κινούνται κάθετα και παράλληλα στο μαγνητικό πεδίο
                %, δηλαδή στην περίπτωσή μας $\bm{v_i}= v_{i,x}\hat{x}+v_{i,y}\hat{y} +v_{i,z}\hat{z}$ 
                και επίσης πως η συχνότητά του είναι πολύ μικρότερη από την συχνότητα των κυμάτων Alfven που διαδίδονται παράλληλα στο μαγνητικό πεδίο, $\omega << k_zv_A$. %\textcolor{red}{Εν γένει θα θεωρήσουμε πως η πυκνότητα ισορροπίας $n_0$ έχει μικρή χωρική εξάρτηση, δηλαδή πως το πλάσμα είναι σχεδόν ομογενές στην κατάσταση ισορροπίας του, προκειμένου να μετατίθεται με οποιονδήποτε χωρικό διαφορικό τελεστή.}
	  				
	  		Σε πρώτη φάση θα πρέπει να εξαχθούν οι Γραμμικές Εξισώσεις με βάση τις οποίες θα εξετάσουμε την διάδοση των κυμάτων ολίσθησης. Θα ξεκινήσουμε από την εξίσωση της ορμής, θα διαταρράξουμε σε πρώτη τάξη τα μεγέθη από την ισορροπία, σύμφωνα με τις σχέσεις $n_i = n_0+n_1, p_i=p_0+p_1$ και $\bm{v_i}=\bm{v_0}+\bm{v_1}$, έπειτα θα κάνουμε το ίδιο για την εξίσωση της συνέχειας και εν τέλει βασιζόμενοι σε παραδοχές και προσεγγίσεις για την φύση του υπό μελέτη πλάσματος θα προχωρήσουμε στον συνδιασμό των δύο εξισώσεων με στόχο την απλοιφή της ταχύτητας των ιόντων.

                    Για τα παραπάνω ισχύουν επίσης ότι η ταχύτητα ισορροπίας είναι σχεδόν μηδενική \textcolor{red}{$\bm{v_{0,i}}\simeq0$} και πως όλα τα μεγέθη στην ισορροπία εχουν μηδενική χρονική παράγωγο. Ο λόγος για τον οποίο η ταχύτητα των ιόντων στην ισορροπία είναι περίπου μηδενική και όχι ακριβώς μηδενική έγκειται στο ότι υπάρχει η διαμαγνητική ταχύτητα ολίσθησης. Ωστόσο, αυτή είναι ανάλογη της θερμοκρασίας των ιόντων $\sim T_i$ την οποία θεωρούμε πολύ μικρή. Ακριβώς σε αυτό το σημείο ίσως μπορούμε να εντοπίσουμε και έναν από τους λόγους για τους οποίους ''γίνεται η επιλογή'' να χρησιμοποιηθούν οι εξισώσεις των ιόντων. Αν δουλεύαμε με αυτές των ηλεκτρονίων, στην εξίσωση συνέχειας θα είχαμε έναν επιπλέον όρο $\bm{u_{0,e}}\nabla n_1$, ενώ στην ορμής θα είχαμε τους $\bm{u_{0,e}}\cdot\nabla \bm{u_{1,e}}$ και $\bm{u_{0,e}}\times\bm{B_1}$


                \subsubsection*{Εξίσωση Ορμής}
                    Η εξίσωση της ορμής για τα ιόντα στην θεώρηση των δύο συζευγμένων ρευστών είναι 
                        \begin{align*}
                            m_in_i\frac{D\bm{v_i}}{Dt} = n_i q_i (\bm{E}+\bm{u_i}\times\bm{B})-\nabla p_i 
                        \end{align*}
                        
                    Επειδή έχουμε υποθέσει εξ' αρχής πως το πλάσμα έχει χαμηλό $\beta$, δηλαδή ότι η κινητική πίεση είναι πολύ μικρότερη της μαγνητικής, μπορούμε να αγνοήσουμε τον όρο πίεσης στην παραπάνω εξίσωση.
                    Aντικαθιστούμε τις διαταραγμένες τιμές των μεγεθών 
                        \begin{align*}
                            m_i (\partial_t + \bm{v_{i,1}}\cdot \nabla)\bm{v_{i,1}} = q_i(\bm{E_0}+\bm{E_1}) + q_i\bm{v_{i,1}}\times(\bm{B_0}+\bm{B_1})  \hspace{2cm} %\textcolor{green}{\star\star\star}
                        \end{align*}
                        
                    Έτσι, σε πρώτη τάξη έχουμε 
                        \begin{align}\label{eq8}
                            m_i\partial_t\bm{u_{i,1}}  = q_i \bm{E_1} + q_i\bm{u_{i,1}}\times\bm{B_0}    
                        \end{align}

                    Επειδή πρόκειται για ηλεκτροστατικές ταλαντώσεις, τόσο το ηλεκτρικό πεδίο $\bm{E_1}$ όσο και η ταχύτητα   $\bm{u_{i1}}$ θα είναι ανάλογα του $\sim e^{-i\omega t}$, έτσι η παραπάνω εξίσωση γίνεται 
                        \begin{align*}\label{eq9}
                            -i\omega m_i\bm{u_{i,1}}  =& q_i \bm{E_1} + q_i\bm{u_{i,1}}\times\bm{B_0}\Rightarrow\\
                            %%%
                            \bm{u_{i,1}} =& -\frac{q_i}{i\omega m_i}\bm{E_1} + \bm{u_{i,1}}\times\left(-\frac{q_i}{i\omega m_i}\bm{B_0}\right) \numberthis
                        \end{align*}

            \subsubsection*{Εξίσωση Συνέχειας}
            Η Εξίσωση Συνέχειας για τα ιόντα είναι 
                \begin{align*}
                    \partial_tn_i + \nabla(\bm{u_i}n_i) = 0 
                \end{align*}
            για τις συνηθισμένες διαταρραχές των μεγεθών γίνεται 
                \begin{align*}
                    \partial_t n_{i,0}+\partial_t n_{i,1}  + \nabla(\bm{u_{i,1}} n_{i,0}  + \bm{u_{i,1}} n_{i,1}) =0
                \end{align*}
                Κρατώντας μόνο τους όρους πρώτης τάξης παίρνουμε 
                    \begin{align}\label{eq10}
                        \partial_t n_{i,1} + n_{i,0}\nabla\bm{u_{i,1}} + \bm{u_{i,1}}\nabla n_{i,0} = 0 
                    \end{align}
        

        \subsubsection*{Συνδυασμός των Δύο Εξισώσεων και Εισαγωγή των Παραδοχών-Προσεγγίσεων}
            
		Παρατηρώ ότι από τις εξισώσεις (\ref{eq9}) και (\ref{eq10}) δεν γνωρίζω μόνο την ταχύτητα και το ηλεκτρικό πεδίο. Έτσι, αν γίνει απολοιφή της ταχύτητας θα απομείνει μόνο το ηλεκτρικό πεδίο το οποίο θα αντιμετωπιστεί μετέπειτα.            
            
                    Αρχικά βλέπουμε πως η σχέση (\ref{eq9}) που προκύπτει από την εξίσωση ορμής των ιόντων είναι μία αλγεβρική εξίσωση της μορφής 
                        \begin{align}\label{eq11}
                            \bm{u_{i,1}} = \bm{a} + \bm{u_{i,1}}\times\bm{b}
                        \end{align}
                        οπου $\bm{a} = -\frac{q_i}{i\omega m_i}\bm{E_1}=-\frac{\Omega_{i}}{i\omega B_0}\bm{E_1}$ και $\bm{b} = -\frac{q_i}{i\omega m_i}\bm{B_0} = - \frac{\Omega_i}{i\omega B_0}\bm{B_0}$, όπου $\Omega_i = \frac{q_iB_0}{m_i}$ η κυκλοτρονική συχνότητα των ιόντων. Μία τέτοια εξίσωση μπορεί να επιλυθεί ως προς την ταχύτητα $\bm{u_{i1}}$, αλλά με έναν όχι και τόσο ευθύ τρόπο. Αφού την λύσω, θα αντικαταστήσω το αποτέλεσμα στην Εξίσωση Συνέχειας (\ref{eq10}) και θα εισάγω τις κατάλληλες προσεγγίσεις. 

                        Γνωρίζουμε από την Γραμμική Άλγεβρα πως οποιοδήποτε διάνυσμα στον τρισδιάστατο χώρο μπορεί να γραφεί ως Γραμμικός Συνδυασμός τριών Γραμμικά Ανεξάρτητων διανυσμάτων. Στην περίπτωσή μας ισχύουν $\bm{a}\perp\bm{a}\times\bm{b}$ και $\bm{b}\perp\bm{a}\times\bm{b}$, άρα τα $\bm{a},\bm{b}$ είναι Γραμμικώς Ανεξάρτητα με το $\bm{a}\times\bm{b}$. Ακόμη, το $\bm{b}\parallel\bm{b_0}\parallel\hat{z}$,  ενώ το $\bm{a}\parallel\bm{E_1}$ θα έχει και άλλες συνιστώσες, όπως για παράδειγμα την y που είναι κάθετη στην ανομοιογένεια της πυκνότητας, επομένως και τα $\bm{a}, \bm{b}$ είναι Γραμμικώς Ανεξάρτητα. Άρα το διάνυσμα της ταχύτητας γράφεται ως 
                            \begin{align}\label{eq12}
                                \bm{u_{i1}} = A\bm{a} + B\bm{b} + C (\bm{a}\times\bm{b}) 
                            \end{align}

                        Τώρα, αντικαθιστώ την (\ref{eq12}) στην (\ref{eq11}) προκειμένου να πάρω τους συντελεστές A,B,C

                            \begin{align*}
                                \underline{A\bm{a}} + \textcolor{red}{B\bm{b}} + \textcolor{green}{C (\bm{a}\times\bm{b})} =& \bm{a} + A\bm{a}\times\bm{b} + B\cancel{\bm{b}\times\bm{b}} + C(\bm{a}\times\bm{b})\times\bm{b} \Rightarrow\footnotemark\\ 
                                %
                                =&\underline{\bm{a}} + \textcolor{green}{A\bm{a}\times\bm{b}} + \textcolor{red}{C\bm{b}(\bm{a}\cdot\bm{b})} - C \underline{\bm{a}(\bm{b}\cdot\bm{b})}
                            \end{align*}
                                \footnotetext{Χρησιμοποιώ της ταυτότητα $(\bm{B}\times\bm{C})\times\bm{A} = \bm{C}(\bm{A}\cdot\bm{B})-\bm{B}(\bm{A}\cdot\bm{C})$}

                        Δεδομένου ότι τα διανύσματα $\bm{a},\bm{b},\bm{a}\times\bm{b}$ είναι Γραμμικώς Ανεξάρτητα, οι αντίστοιχοι συντελεστές τους στα δύο μέλη της παραπάνω εξίσωσης θα πρέπει αναγκαστικά να είναι ίδιοι, άρα 

                           % \begin{align*}
                            %    \left. A = 1 - Cb^2\\
                             %          B = C (\bm{a}\cdot\bm{b})\\ 
                              %         C = A \right\} \Rightarrow
                            %\end{align*}

                            
                            \begin{equation*}
                                  \left.\begin{aligned}
                                  A =& 1 - Cb^2\\
                                  B =& C (\bm{a}\cdot\bm{b})\\ 
                                  C =& A
                                \end{aligned}\right\} \Rightarrow \left\{ \begin{aligned}
                                    A =& C = \frac{1}{1+b^2}\\
                                    C =& \frac{\bm{a}\cdot\bm{b}}{1+b^2}
                                \end{aligned}\right.
                            \end{equation*}

        Άρα η ταχύτητα είναι 
            \begin{align*}
                \bm{u_{i1}} =& \frac{1}{1+b^2}\left( \bm{a}+(\bm{a}\cdot\bm{b})\bm{b}+\bm{a}\times\bm{b}\right) \xRightarrow{\bm{a} = -\frac{\Omega_{i}}{i\omega B_0}\bm{E_1}, \bm{b} = - \frac{\Omega_i}{i\omega B_0}\bm{B_0}, \bm{B_0}=B_0\hat{z}}\\
                %
                    =& \frac{1}{1-(\Omega_{i}/\omega)^2}\left( -\frac{\Omega_{i}}{i\omega B_0}\bm{E_1} + \frac{\Omega_{i}^2}{\omega^2 B_0^2} (\bm{E_1}\cdot\bm{B_0})\frac{\Omega_{i}}{i\omega B_0}\bm{B_0} + \frac{\Omega_{i}}{i\omega B_0}\frac{\Omega_{i}}{i\omega B_0}\bm{E_1}\times\bm{B_0}  \right) \Rightarrow \\
                    =& \frac{1}{1-(\Omega_{i}/\omega)^2} \frac{\Omega_i}{i\omega B_0}\left( -\bm{E_1} + \frac{\Omega_i^2}{\omega^2 B_0^2}(\bm{E_1}\cdot\bm{B_0})\bm{B_0}+\frac{\Omega_i}{i\omega B_0}\bm{E_1}\times\bm{B_0}\right)\Rightarrow\\
                    =& Q\left( -\bm{E_1} + \frac{\Omega_i^2}{\omega^2 B_0^2}(\bm{E_1}\cdot\bm{B_0})\bm{B_0}+\frac{\Omega_i}{i\omega B_0}\bm{E_1}\times\bm{B_0}\right)
            \end{align*}
            όπου $Q=\frac{1}{1-(\Omega_{i}/\omega)^2} \frac{\Omega_i}{i\omega B_0}$. Αντικαθιστώ τώρα την έκφραση της ταχύτητας στην Εξίσωση Συνέχειας (\ref{eq10})
                \begin{align*}
                    \partial_t n_{i,1} +& n_{i,0}\nabla\bm{u_{i,1}} + \bm{u_{i,1}}\nabla n_{i,0} = 0 \Rightarrow\\
                    -\partial_tn_{i,1}= &n_{i,0}Q\left( -\nabla\bm{E_1} + \frac{\Omega_i^2}{\omega^2 B_0^2}\nabla\left[(\bm{E_1}\cdot\bm{B_0})\bm{B_0}\right]+\frac{\Omega_i}{i\omega B_0}\nabla\left[\bm{E_1}\times\bm{B_0}\right]\right) +\\ 
                    &\textcolor{white}{asafssfadsfadsfadsf }+ Q\left( -\bm{E_1} + \frac{\Omega_i^2}{\omega^2 B_0^2}(\bm{E_1}\cdot\bm{B_0})\bm{B_0}+\frac{\Omega_i}{i\omega B_0}\bm{E_1}\times\bm{B_0}\right)\cdot\nabla n_{i,0}
                \end{align*}
                
                Σε αυτό το σημείο θα εισάγω κάποιες παραδοχές και επίσης θα απαλοίψω το ηλεκτρικό πεδίο. Έχουμε υποθέσει πως τα κύματα ολίσθησης είναι χαμηλής συχνότητας άρα θα ισχύει ότι $\omega/\Omega_i <<1$ και γι' αυτό γίνεται η εξής απλοποίηση 
                	\begin{align*} 
                		Q = \frac{1}{1-(\Omega_{i}/\omega)^2} \frac{\Omega_i}{i\omega B_0} \simeq -\frac{\omega^2}{\Omega_i^2}\frac{\Omega_i}{i\omega B_0} =-\frac{\omega}{i\Omega_iB_0}
                	\end{align*}
                	
                Ακόμη, παρατηρούμε πως το μαγνητικό πεδίο πρώτης τάξης $\bm{B_1}$ δεν εμφανίζεται στην παραπάνω εξίσωση, άρα δεν θα έχει σημαντική επίδραση στην δυναμική εξέλιξη των κυμάτων, και γι' αυτό δικαιούμαστε να θεωρήσουμε ηλεκτροστατικές ταλαντώσεις όπως έχουμε ήδη κάνει. 
                Γράφουμε $\bm{E_1} = -\nabla \Phi_1$. Αυτό το βήμα θα απλοποιήσει σημαντικά το δεξί μέλος της εξίσωσης. Επίσης, η κλιση του Ηλεκτρικού Δυναμικού είναι κάθετη στη διεύθυνση κατά την οποία μεταβάλλεται η πυκνότητα ισορροπίας. Με αυτές τις παραδοχές το δεξί μέλος της παραπάνω εξίσωσης γίνεται
                \begin{itemize}
                 	\item[.] \underline{1ος όρος:} \hspace{0.5cm} $-\nabla\bm{E_1} = \nabla^2\Phi_1$
                 	 
                	\item[.] \underline{2ος όρος:} \hspace{0.5cm}$\nabla \left[(\bm{E_1}\cdot\bm{B_0})\bm{B_0}\right] = - \nabla \left[(\nabla\Phi_1\cdot\bm{B_0})\bm{B_0}\right] = -Β_0^2 \nabla_{\parallel}^2\Phi_1$ 
                	\item[.] \underline{3ος όρος:} \hspace{0.5cm} $\nabla\left(\bm{E_1}\times\bm{B_0}\right) = - \nabla\left(\nabla\Phi_1\times\bm{B_0})\right) = -\bm{B_0}\cdot\cancelto{Εξ'ορισμου}{(\nabla \times(\nabla\Phi_1))} + \nabla\Phi_1\cdot\cancelto{Σταθερo}{(\nabla\times\bm{B_0})} = 0 $
                	\item[.] \underline{4ος όρος:} \hspace{0.5cm}
                			$-\bm{E_1}\cdot\nabla n_{i,0} = \nabla\Phi_1\cdot\nabla n_{i,0} = 0 $, διότι $\nabla\Phi_1\perp\nabla n_{i,0}$
                	\item[.] \underline{5ος όρος:} \hspace{0.5cm}
                		$(\bm{E_1}\cdot\bm{B_0})\bm{B_0}\cdot\nabla n_{i,0} = - (\nabla\Phi_1\cdot\bm{B_0})\bm{B_0}\cdot\nabla n_{i,0} = - B_0\left(\nabla\Phi_1\right)_{\parallel}\bm{B_0}\cdot\nabla n_{i,0} = -B_0^2\nabla_{\parallel}\Phi_1\cdot\nabla n_{i,0}=0$
                	\item[.] \underline{6ος όρος:} \hspace{0.5cm}
                	$(\bm{E_1}\times\bm{B_0}) \cdot \nabla n_{i,0} = -B_0(\nabla\Phi_1\times \hat{z}) \cdot\nabla n_{i,0} = -B_0(\partial_y\Phi_1\hat{x} - \partial_x\Phi_1\hat{y})\cdot (\partial_x n_{i,0} \hat{x})  = -B_0 (\partial_xn_{i,0})(\partial_y\Phi_1)$
                \end{itemize}
                
     Άρα αν αντικαταστήσουμε τα παραπάνω στην εξίσωσή μας παρίνουμε 
             	\begin{align}\label{eq13}
             		-\partial_t n_{i,1} = -\frac{\omega n_{i,0}}{i\Omega_iB_0} \nabla^2\Phi_1 
             		  +\frac{\Omega_in_{i,0}}{iB_0\omega }\nabla_{\parallel}^2\Phi_1   
             		  -\frac{1}{B_0}(\partial_xn_{i,0})(\partial_y\Phi_1)
				\end{align}
				
	Οι όροι του δυναμικό όπως και της πυκνότητας πρώτης τάξης θα είναι ανάλογοι $\sim exp(-i\omega t)$, επομένως αν κάνουμε μία τέτοια αντικατάστασταση θα προκύψει η παρακάτω εξίσωση 
	\begin{align*}\label{eq14}
		i\omega n_{i,1} =& -\frac{\omega n_{i,0}}{i\Omega_iB_0} \nabla^2\Phi_1 +\frac{\Omega_in_{i,0}}{iB_0\omega }\nabla_{\parallel}^2\Phi_1-\frac{1}{B_0}(\partial_xn_{i,0})(\partial_y\Phi_1) \Rightarrow\\ 
		n_{i,1} =&+ \frac{n_{i,0}}{\Omega_i B_0}\nabla^2\Phi_1 - \frac{\Omega_in_{i,0}}{B_0\omega^2	}\nabla_{\parallel}^2\Phi_1 - \frac{1}{i\omega B_0}(\partial_xn_{i,0})(\partial_y\Phi_1) \numberthis
	\end{align*}
	
		Έως αυτό το σημείο παρατηρούμε πως έχουν απομείνει δύο όροι πρώτης τάξης τους οποίους πρέπει με κάποιον τρόπο να απαλοίψουμε προρκειμένου να προκύψει μία εξίσωση ως προς το $\omega$. δηλαδή η σχέση διασποράς. Θα χρησιμοποιήσουμε δύο ακόμη παραδοχές σχετικά με την πυκνότητα των ηλεκτρονίων και των ιόντων. Η πρώτη παραδοχή σχετίζεται με το γεγονός ότι θέλουμε να διατηρείται η μακροσκοπική ουδετερότητα του πλάσματος (Quasineutrality), δηλαδή οι πυκνότητες ηλεκτρονίων και ιόντων θα πρέπει να είναι ίσες
		\begin{subequations}\label{eq15}
		\begin{align}
			n_{i,1} = n_{e,1} \\ 
			n_{i,0} = n_{e,0}
		\end{align}
		\end{subequations}

%%%%% νευτρανιλτυ
	\noindent\fbox{	\parbox{\textwidth}{ \begin{addmargin}[2cm]{2cm}
 	\subsubsection*{\textcolor{darkgray}{Αιτιολόγηση της Μακροσκοπικής Ουδετερότητας [5]}}
  		
	 \textcolor{darkgray}{		Σε αυτό το σημείο θα αιτολογηθεί η χρήση του επιχειρήματος της Μακροσκοπικής Ουδετερότητας, δηλαδή ότι $n_i = n_e$.%, που χρησιμοποιήθηκε για την εξαγωγή της σχέσης διασποράς των κυμάτων ολίσθησης.
			Εφόσον έχουμε ηλεκτροστατικές ταλαντώσεις θα ισχύει η εξίσωση Poisson
			\begin{align*}
				\nabla^2\Phi_1 = - 4\pi q(n_{i,1}-n_{e,1})	
			\end{align*}			  			
  		Αν δουλέυουμε σε κλίμακα μεγαλύτερη του μήκους Debye $\lambda_D = (k_BT_e/4\pi Nq^2)^{1/2}$, θα έχουμε 
  		\begin{align*}
  			\frac{n_{i,1}-n_{e,1}}{N} = - \frac{k_BT_e}{4\pi Nq^2}\nabla^2\left( \frac{q\Phi}{k_BT_e}\right) \simeq \frac{\lambda_D^2}{\delta x^2}\frac{q\Phi}{k_BT_e}<<1
  		\end{align*}
  	Άρα προκύπτει ότι σε χωρική κλίμακα πολύ μεγαλύτερη του μήκους Debye ισχύει $n_{e_1} \simeq n_{i,1}$}\newline\textcolor{white}{gg}
  		\end{addmargin}} 	}		
		
		%%%%%%
		%%%%%%
		\newpage
		
	Ακόμη, εφόσον γνωρίζουμε πως η πυκνότητα ισορροπίας των ηλεκτρονίων ακολουθεί την στατιστική Boltzmann, αν αναπτύξουμε κατά Taylor θα γίνει 
		\begin{align*}
			n_e = n_{e,0} + n_{e,0} \frac{q\Phi_1}{k_BT_e} + O(\cdot^2)
		\end{align*}
	Η πυκνότητα πρώτης τάξης είναι 
% \footnote{Αυτό προκύπτει και από την εξίσωση ορμής των ηλεκτρονίων στην ι}
		\begin{align}\label{eq16}
			n_{e,1} = n_{e,0} \frac{q\Phi_1}{k_BT_e}
		\end{align}
	Τώρα, από την σχέση (\ref{eq15}), αντικαθστώντας με τις (\ref{eq14}) και (\ref{eq16}) έχουμε
		\begin{align*}
			n_{e,1} = n_{i,1} \Rightarrow\\ 
            \left( \frac{q_e}{k_B T_e} - \frac{1}{\Omega_i B_0}\nabla^2 + \frac{\Omega_i}{B_0\omega^2	}\nabla_{\parallel}^2 + \frac{1}{i\omega B_0}\frac{\partial_xn_{i,0}}{n_{i,0}}\partial_y    \right)\Phi_1   = 0 \Rightarrow\\
           \left( 1 - \frac{k_BT_e}{\Omega_i B_0q}\nabla^2 +  
           		 \frac{\Omega_ik_BT_e}{B_0\omega^2q}\nabla_{\parallel}^2 + \frac{k_BT_e}{i\omega B_0 q}\frac{\partial_xn_{i,0}}{n_{i,0}}\partial_y    \right)\Phi_1  	 =0 	\Rightarrow 
		\end{align*}
	Αν αντικαταστήσουμε το δυναμικό ως $\Phi_1 = \Phi_1(t) e^{i\bm{k}\cdot\bm{r}}$, με $\bm{k} = k_y \hat{y} + k_z\hat{z}$ θα παρουμε 
	\begin{align*}
		1 + \frac{k_BT_e}{\Omega_i B_0q}k^2 -  
           		 \frac{\Omega_ik_BT_e}{B_0\omega^2q}k_z^2 + \frac{k_BT_e}{\omega B_0 q}\frac{\partial_xn_{i,0}}{n_{i,0}}k_y =0  
	\end{align*}
	%%%%%%%%%%%%%%%%%%%%%%%%%%%%%%%%55
	Γνωρίζουμε ότι $v_s^2 = k_BT_e/m_i$, $\Omega_i = q_iB_0/m_i$ και $v_{D,e} = -(k_BT_e/qB_0)(\partial_xn_{e,0}/ n_{e,0})$, άρα 
	\begin{align*}\label{eq17}
		1 + \frac{v_s^2}{\Omega_i^2}k^2 - \frac{v_s^2k_z^2}{\omega^2} - \frac{v_{D,e}k_y}{\omega} = 0 \Rightarrow \\ 
  		1 + \left( \frac{\Omega_i}{u_sk}\right)^2\left(\frac{\omega^2 - (u_sk_z)^2 - (u_{D,e}k_y)\omega}{\omega^2}	\right) =0\numberthis
  			\end{align*}
  	%%%%%%%%%%%%%%%%%	
  		Η σχέση  (\ref{eq17}) είναι και η ζητούμενη σχέση διασποράς για τα κύματα ολίσθησης.
  		
  		\subsubsection*{Τελικές Παρατηρήσεις}

Αν θεωρήσουμε $r=u_s/\Omega_i$, τότε η σχέση (\ref{eq17}) γίνεται

\begin{align*}
    (1+(kr)^2)\omega^2 -(u_{D,e}k_y)\omega -(k_zu_s)^2 =0 
\end{align*}
%Δεδομένης της σχέσης (\ref{eq5}) μπορούμε να θεωρήσουμε ότι το κύμα μας διαδίδεται περισσότερο προς την κατεύθυνση y, δηλαδή ότι  $k_y>>k_z\simeq 0$. Τότε η παραπάνω σχέση θα γίνει 
  %  \begin{align}
   %     (1+(k_yr)^2)\omega - u_{D,e}^2k_y \simeq0
  %  \end{align}
%   Επομένως όταν $\omega < u_{D,e}k_y/(1+(k_yr)^2)$ το κύμα θα είναι ασταθές. Η εν λόγω αστάθεια οφείλεται στις συγκρούσεις ηλεκτρονίων-ιόντων.
  Παρατηρώ ότι $\Delta = (u_{D,e}k_y)^2+4(1+(kr)^2)(k_zu_s)^2 >0$, άρα $Re(\omega)>0$ και έτσι με την παραπάνω σχέση διασποράς δεν προβλέπονται φαινόμενα απόσβεσης των κυμάτων ολίσθησης.
  
  Παρόμοια σχέση διασποράς (\ref{eq17}) μπορεί να προκύψει και από την κινητική θεωρία συνδυάζοντας την εξίσωση Vlasov με την εξίσωση Poisson για το Ηλεκτρικό Δυναμικό και τις κατάλληλες προσεγγίσεις για τα διάφορα μεγέθη όπως έγιναν και στην ανάλυση με την Θεωρία των δύο ρευστών. Από την κινητική θεωρία προκύπτει ένας επιπλέον όρος στην Σχέση Διασποράς ο οποίος έχει φανταστικό μέρος. Ο όρος αυτός σχετίζεται με το φαίνόμενο Landau Damping, δηλαδή στην εξασθένιση (ή ενίσχυση) του κύματος λόγω της αλληλεπίδρασής του με συγκεκριμένα σωματίδια. Μερικά από τα σωματίδια του πλάσματος έχουν ταχύτητες οι οποίες είναι πολύ κοντά στην φασική ταχύτητα του κύματος, είτε λίγο μεγαλύτερες είτε λίγο μικρότερες. Εξαιτίας αυτού, συντονίζονται με το κύμα και η ανταλλαγή ενέργειας κύματος-σωματιδίων είναι η βέλτιστη δυνατή. Σε όσα σωματίδια έχουν μικρότερη ταχύτητα από την φασική, εναποτίθεται ενέργεια από το κύμα, ενώ όσα έχουν μεγαλύτερη, μεταφέρουν την κινητική τους ενέργεια στο κύμα. Στην περίπτωση των κυμάτων ολίσθησης, μπορεί να έχουμε τόσο απόσβεση όσο και ενίσχυση.
  				
  			
  	
  	
  %	\subsubsection*{Further}
  	
  	%		Drift Wave Turbulence near edge of tokamak Hasegawa - Wakatani 
  			\subsection*{Βιβλιογραφία}
  				\begin{itemize}
  					\item[[1]] F. F. Chen - "Introduction to Plasma Physics and Controlled Fusion: Volume 1: Plasma Physics" 
  					\item[[2]] Λ. Βλάχος: "Φυσική Πλάσματος"
  					\item[[3]] R. Goldston - P. H. Rutherford: "Introduction to Plasma Physics"
  					\item[[4]] W. A. Strauss: "Partial Differential Equations: An Introduction"
  					\item[[5]] W. Horton: "Drift Waves and Transport", Rev. Mod. Phys. 71, 735 
  				\end{itemize}
\end{document}
