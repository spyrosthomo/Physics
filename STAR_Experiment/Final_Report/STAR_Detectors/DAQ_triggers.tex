\section{Data Acquisition System (DAQ) \& Triggers}

Ο ρόλος του Data Qcquuisition System (DAQ) είναι να διαβάσει τα δεδομένα που συλλέγονται από τους υποανιχνευτές του STAR και να τα μειώσε σε διαχειρήσιμα επίπεδα. Τα σημεία στα οποία λαμβάνονται οι αποφάσεις για το ποιά γεγονότα είναι κρίσιμα καλούνται \textit{triggers} και βασίζονται τόσο σε \textit{hardware}  όσο και σε \textit{software}. 
	Το \textit{trigger} γίνεται σε 4 επίπεδα. Τα πρώτα δύο 0 και 1 βασίζονται σε \textit{hardware} καθώς φιλτράρουν γεγονότα με μεγαλύτερη συχνότητα και πρέπει να αποκλείουν τα μη ενδιαφέροντα με μεγάλη ταχύτητα σε αντίθεση με τα 2 και 3 τα οποία βασίζονται σε \textit{software}.
	
	Κάποιες αποφάσεις στο επίπεδο 0 λαμβάνονται με βάση την ενέργεια που ενοποτίθεται σε ορισμένες περιοχές των BEMC \& EEMC. Έπειτα, τα δεδομένα των γεγονότων που περνάνε το πρώτο τεστ οδηγούνται στο επίπεδο 2 το οποίο πρόκειται για ένα πρόγραμμα που φιλτράρει μόνο τα γεγονότα ηλεκτρονίων ή φωτονίων. Αν κάποιο γεγονός περάσει και από το επίπεδο 2, τότε τα δεδομένα από όλα τα υποσυστήματα του STAR που σχετίζονται με αυτό το γεγονός θα οδηγηθούν στο σύστημα DAQ.