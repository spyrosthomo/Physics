\section{Endcap Electromagnetic Calorimeter (EEMC)}

	Το Endcap Electromagnetic Calorimeter (EEMC) παρέχει κάλυψη σε γωνία $\Delta\phi=2\pi$ και $1.086<\eta<2.000$ για υψηλής $p_T$ φωτόνια, ηλεκτρόνια και διασπώμενα μεσόνια. 
	Όπως και το BEMP περιέχει έναν SMD (Shower Maximum Detector) για διάκριση των $\pi^0/\gamma$, preshower-postshower στρώματα για την διάκριση ηλεκτρονίων - φορτισμένων αδρονίων. Η συνεισφορά του είναι μεγαλύτερη στο πρόγραμμα για τα πολωμένα πρωτόνιο παρά για τα φορτισμένα ιόντα. Εκεί ένας σημαντικός στόχος είναι ο προσδιορισμός της προτιμώμενης ελικότητας των γλουονίων εντός ενός πολωμένου πρωτονίου σαν συνάρτηση του ποσοστού της ορμής ($x_g$) που κατέχει. Επίσης, στόχος του προγράμματος πολωμένων πρωτονίων είναι η συνολική συνεισφορά του σπιν των γλουονίων στο σπιν του πρωτονίου.
	Η πόλωση των γλουονίων μπορεί να ανιχνευτεί σε σκεδάσεις Compton κουαρκ-γλουονίων, μετρώντας τις διαμήκεις συσχετίσεις σπιν. 
	Ακόμη, το EEMC θα βελτιώσει την ευαισθησία του STAR στην εξάρτηση της γεύσης της πόλωσης των τελικών κουάρκ μέσω της παραγωγής $W^{\pm}$.	
	H εστίαση στις κρούσεις p-p συνεπάγεται πως ειναι δυνατή η μείωση των αριθμών των ξεχωριστών κομματιών, σε αντίθεση με το BEMC που εστίαζε στις κρούσεις Au-Au που παράγουν πολλά σωματίδια, άρα η μεγαλύτερη διαμέρισή του ήταν πιό αναγκαία.  
	
	Ομοίως με το BEMC, το εν λόγω καλορίμετρο αποτελείται από διαδοχικά στρώματα μολύβδου-πλαστικού σπινθηριστή (23 και 24 αντίστοιχα) που επιλέχθηκαν λόγω χαμηλού κόστους. Συνολικά καλύπτει περίπου 21.4$X_0$ και λιγότερο από ένα interaction length $\lambda$.
	Οπτικές ίνες συνδέουν τις μονωμένες από φως περιοχές του σπινθηριστή με τους φωτοπολλαπλασιαστές που βρίσκονται και πάλι εκτός της περιοχής ανίχνευσης.
	
	Ο SMD τοποθετείται μετά το πέμπτος ζεύγος μολύβδου-σπινθηριστή και είναι χρήσιμος για την διάκριση μεταξύ ηλεκτρονίων/αδρονίων καθώς και την αντιστοίχιστη των $e^\pm$ με εκείνα που ανιχνεύονται στον TPC. Εδώ δεν πρόκειται για θάλαμο αερίου αλλά αποτελείται και αυτός απο σπινθηριστή που παράγει ένα οπτικό σήμα οδηγούμενο προς τους φωτοπολλαπλασιαστές.
	
	Υπάρχει επίσης και ο pre-shower ανιχνευτής. Αυτός δεν είναι τίποτα άλλο από τα δύο πρώτα στρώματα σπινθηριστή που έχουν συνδεθεί με ξεχωριστή οπτική ίνα. Το οπτικό σήμα αυτών προστίθεται στο σήμα των υπολοίπων σπινθηριστών, ενώ το σήμα από τον δεύτερο καταγραφεται και ξεχωριστά. Έτσι, συγκρίνοντας αυτά τα σήματα με το συνολικό σήμα ενός "πύργου" (δηλαδή μία στοίβας από κάθετα τοποθετημένους σπινθηριστές) μπροούμε να ξεχωρίσουμε τα ηλεκτρόνια από τα αδρόνια εκμεταλλευόμενοι τις διαφορές στην πιθανότητα δημιουργίας καταιγισμού.
	Έπίσης, βελτιώνουν την ικανότητα διάκρισης μεταξύ $\pi^0/\gamma$ καθώς η πιθανότητα ένα ενεργητικό φωτόνιο να αφήσει ενέργεια σε αυτά τα στρώματα είναι διπλάσια.
	
	