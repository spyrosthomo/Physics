\appendix

\chapter{Παράρτημα: Ακτινοβολία Cherenkov}

	Η ακτινοβολία Cherenkov ανιχνεύθηκε πειραματικά για πρώτη φορά το 1934 από τον τότε φοιτητή Cherenkov και ερμηνεύτηκε θεωρητικά το 1937. Πρόκειται για μία ακτινοβολία που προκύπτει από φορτισμένα σματίδια που κινούνται με μεγάλη ταχύτητα εντός ενός διηλεκτρικού μέσου. Δεν εκπέμπεται από τα ίδια τα σωματίδια. Καθώς αυτά περνούν από το διηλεκτρικό το πολώνουν εξαιτίας του ηλεκτρικού τους πεδίου και καθώς αυτά κινούνται με μεγάλη ταχύτητα, το υλικό θέλει να επιστρέψει στην μη πολωμένη κατάστασή του. Κατά την παραπάνω επιστροφή εκπέμπεται η Cherenkov. 
	
	Για να εξετάσουμε μαθηματικά την προέλευσή της θα πρέπει να ξεκινήσουμε από τις εξισώσεις Maxwell σε ένα διηλκεκτρικό μέσο. Ως φορτίο θα βάλουμε ένα κινούμενο σωματίδιο $\rho = e\delta(\bm{r}-\bm{v}t)$,το οποίο ως συνέπεια της κίνησής τους προκαλεί και ένα ηλεκτρικό ρεύμα$\bm{j} = e\bm{u}\delta(\bm{r}-\bm{v}t)$.
	
	\begin{align}
   		\nabla(\hat{\epsilon} \bm{E}) =& 4\pi e\delta(\bm{r}-\bm{v}t) 	 \\ 
   		\nabla\times\bm{E}            =& -\frac{1}{c}\pdv{\bm{H}}{t}  \\ 
   		\nabla \cdot \bm{H}           =& 0 \\ 
   		\nabla\times\bm{H}            =& \frac{1}{c}\pdv{\hat{\epsilon}\bm{E}}{t}+\frac{4\pi}{c}e\bm{u}\delta(\bm{r}-\bm{v}t)
	\end{align}

Θα μας φανούν χρήσιμα τα δυναμικά που αντιστοιχούν στα πεδία 

	\begin{align}	
			\bm{H}=&\nabla\times \bm{A} \\ 
			\bm{E}=&-\frac{1}{c}\pdv{\hat{\epsilon}\phi}{t}-\nabla \phi			
	\end{align}
	
	Επίσης, οι εξισώσεις Maxwell είναι αναλλοίωτες κάτω από μετασχηματισμούς βαθμίδας για τα δυναμικά τους. Άρα μπορούμε να επιλέξουμε αυθαίρετα μία σχέση μεταξύ τους. Επιλέγουμε να δουλέψουμε στην βαθμίδα Lorentz 
	
	\begin{align*}
		\nabla\bm{A} + \frac{1}{c}\pdv{\hat{\epsilon}\phi}{t} = 0 \numberthis
	\end{align*}
	
Αντικαθιστώντας στην  (A.4) τις  (A.5) και (A.6) έχουμε 
 	\begin{align*}
 		\nabla\times\nabla\times\bm{A}  = &-\frac{1}{c^2}\partial_{tt}\bm{A} - \frac{1}{c} \nabla(-c\nabla\bm{A})+\frac{4\pi}{c}e\bm{u}\delta(\bm{r}-\bm{u}t) \Rightarrow\footnotemark\\ 
 		%
 		\textcolor{blue}{\nabla(\nabla\bm{A})}- \nabla^2\bm{A}= &-\frac{1}{c^2}\partial_{tt}\bm{A} -  \textcolor{blue}{\nabla(\nabla\bm{A})}+\frac{4\pi}{c}e\bm{u}\delta(\bm{r}-\bm{u}t) \Rightarrow\\
 		\nabla^2 \bm{A} - \frac{\hat{\epsilon}}{c^2}\pdv[order={2}]{\bm{A}}{t} =& -\frac{4\pi}{c}e\bm{u}\delta(\bm{r}-\bm{u}t) \numberthis
 	\end{align*}
\footnotetext{Ταυτότητα $\nabla\times\nabla\times\bm{A} = \nabla(\nabla\bm{A})- \nabla^2\bm{A}$}
\newpage
Ομοίως, αντικαθιστώντας στην (A.1) τις $\hat{\epsilon}\cdot(A.6)$, (A.7) παίρνουμε  


	\begin{align*}
		\nabla(-\frac{1}{c}\hat{\epsilon}\pdv{\hat{\epsilon}\phi}{t}-\hat{\epsilon}\nabla \phi	)                         =& 4\pi e\delta(\bm{r}-\bm{v}t) \\\Rightarrow^{(A.7)}\\
		-\frac{1}{c}\hat{\epsilon}\partial_t (-\frac{1}{c}\partial_t(\hat{\epsilon}\phi)) - \hat{\epsilon}\nabla^2\phi  =& 4\pi e\delta(\bm{r}-\bm{v}t) \Rightarrow \\ 	
	\hat{\epsilon}\left( \nabla^2\phi -\frac{\hat{\epsilon}}{c^2}\pdv[order={2}]{\phi}{t}\right)                                    =& 	 -4\pi e\delta(\bm{r}-\bm{u}t)       \numberthis
	\end{align*}

Τώρα, επειδή θέλουμε να μελετήσουμε κύματα, δηλαδή περιοδικά μεταβαλλόμενα πεδία άρα και δυναμικά, θα χρησιμοποιήσουμε αναπτύγματα σε ολοκληρώματα Fourier, δηλαδή 

	\begin{align}
		\phi   =& \frac{1}{(2\pi)^3} \int_{-\infty}^{\infty}   \phi_{\bm{k}} e^{i\bm{k}\cdot\bm{r}} d^3k \\
		\bm{A} =& \frac{1}{(2\pi)^3} \int_{-\infty}^{\infty} \bm{A}_{\bm{k}} e^{i\bm{k}\cdot\bm{r}} d^3k
	\end{align}
 Αρα οι αντίστροφοι μετασχηματισμοί θα είναι 
 	\begin{align}
 		\phi_{\bm{k}} =& \int_{-\infty}^{\infty}   \phi e^{-i\bm{k}\cdot\bm{r}} d^3r \\ 
 		\bm{A_{\bm{k}}} =&  \int_{-\infty}^{\infty} \bm{A} e^{-i\bm{k}\cdot\bm{r}} d^3r
 	\end{align}

Άρα πολλαπλασιάζοντας τις σχέσεις (A.8) \& (A.9) με $e^{-\bm{k}\cdot\bm{r}}$ και ολοκληρώνοντας ως προς $d^3r$, ππρακτικά μετασχιματίζοντας με αντίστροφο Fourier παίρνουμε π.χ. για την (A.9) 
	\begin{align*}
		\hat{\epsilon}\left(\int_{-\infty}^{\infty}\nabla^2\phi e^{-i\bm{k}\cdot\bm{r}}d^3r - \frac{\hat{\epsilon}}{c^2} \pdv[order={2}]{}{t}\int_{-\infty}^{\infty}\phi e^{-i\bm{k}\cdot\bm{r}}d^3r \right) =& -4\pi e\int_{-\infty}^{\infty} \delta(\bm{r}-\bm{u}t)e^{-i\bm{k}\cdot\bm{r}}d^3 r  \xRightarrow{\text{2 παραγοντικές στο 1ο ολοκλ.}}\\
	\hat{\epsilon} \left( -k^2 \phi_{\bm{k}} + \frac{\epsilon}{c^2} \pdv[order={2}]{\phi_{\bm{k}}}{t}   \right) =& - 4\pi e e^{-i\bm{k}\cdot\bm{u}t} \numberthis
	\end{align*}

	Ομοίως για την (A.8) 
	\begin{align*}
		k^2 \bm{A_{\bm{k}}} + \frac{\epsilon}{c^2} \pdv[order={2}]{\bm{A_{\bm{k}}}}{t}  = \frac{4\pi e \bm{u}}{c} e^{-\bm{u}\cdot\bm{k}t} \numberthis
	\end{align*}


Άρα από τις δύο τελευταίες σχέσεις βλέπουμε ότι η εξάρτηση των δυναμικών από τον χρόνο είναι  $\bm{A_{\bm{k}}}\sim e^{-\bm{u}\cdot\bm{k}t}$ και $\phi_{\bm{k}} \sim e^{-\bm{u}\cdot\bm{k}t}$, άρα αντικαθιστώντας αυτές τις εκφάσεις στις παραπάνω σχέσεις και δεδομένου πως $\omega=\bm{k}\cdot\bm{r} $ παίρνουμε

	\begin{align*}
		\hat{\epsilon} \left( -k^2 \phi_{\bm{k}} -\omega^2 \frac{\epsilon}{c^2} \phi_{\bm{k}}   \right) =& - 4\pi e e^{-i\omega t}  \Rightarrow \\
		%
	\phi_{\bm{k}} =& \underbrace{\frac{4\pi e}{\hat{\epsilon}(\omega)}\frac{1}{k^2 -\frac{\hat{\epsilon}(\omega)}{c^2}\omega^2}}_{\text{$\phi_0$}} e^{-\omega t}	\numberthis
	\end{align*}
και  
 \begin{align*}\label{A.17}
 	k^2 \bm{A_{\bm{k}}} - \frac{\hat{\epsilon}}{c^2 }\omega^2 \bm{A_{\bm{k}}} =& \frac{4\pi e\bm{u}}{c} e^{-i\omega t}  \Rightarrow\\ 
 	%
 	\bm{A_{\bm{k}}} =& \underbrace{\frac{4\pi e }{c} \frac{\bm{u} }{k^2 -\frac{\hat{\epsilon}(\omega)}{c^2}\omega^2}}_{\text{$\bm{A_{\bm{k}0}}$}} e^{-i\omega t} \numberthis
 \end{align*}

Οι αντίστοιχοι συντελετστές 	Fourier του ηλεκτρικού πεδίου δίνονται από την βαθμίδα Lorentz,\textcolor{red}{ σχέση (A.7)}
	\begin{align*}
		\bm{E_{\bm{k}}} = \frac{1}{c}i\omega \bm{A_{\bm{k}} }- i \bm{k} \phi_{\bm{k}} \numberthis
	\end{align*}


Τώρα μένει να βρούμε τον μετασχηματισμό Fourier του $\bm{E}$, ώστε από την σχέση $\bm{F} = e\bm{E}$  να βρούμε την δύναμη που ασκείται στο κινούμενο σωματίδιό μας. 
	Άρα έχουμε 
		\begin{align*}
			\bm{E} =& \frac{1}{(2\pi)^3}	\int_{-\infty}^{\infty}   \bm{E_{\bm{k}}} e^{i\bm{k}\cdot\bm{r}}	d^3k \Rightarrow\\
			=& \frac{1}{(2\pi)^3}	\int_{-\infty}^{\infty} \frac{1}{c}i\omega\bm{A_{\bm{k}0}} - i\bm{k}\phi_{\bm{k}0}d^3k \Rightarrow\\
			=& \frac{i}{(2\pi)^3c} \int_{-\infty}^{\infty} \omega\bm{A_{\bm{k}0}}-\bm{k} \phi_{\bm{k}0} c d^k \Rightarrow\\ 
			%%%%%%%%%%
			=& \frac{i}{(2\pi)^3c}4\pi e \int_{-\infty}^{\infty} \frac{\omega}{c} \frac{\bm{u}}{k^2-\omega^2\frac{\hat{\epsilon}(\omega)}{c}} -   \frac{\bm{k}}{\hat{\epsilon}(\omega)} \frac{c}{k^2 - \omega^2\frac{\hat{\epsilon}(\omega)}{c^2}}d^3k \numberthis 
		\end{align*}
		
	Τώρα έχουμε $k_xu_x=\omega$, άρα $dk_x = d\omega /u$ και διαχωρίζουμε τα κυματανύνσματα στην διέθυνση κίνησης από την διεύθυνση κάθετα στην κίνηση θέτωντας  $q=\sqrt{k_y^2+k_z^2}$, άρα $dk_ydk_z=2\pi qdq$ και $k^2=q^2+k_x^2$. Τότε η δύναμη είναι 
	\begin{align*}
		\bm{F} =&  e\bm{E} \xRightarrow{(A.19)} \\ 
			   =& \frac{ie^2}{\textcolor{blue}{2\pi^2}c^2}\int_{-\infty}^{\infty} \frac{1}{q^2+\omega^2\left(\frac{1}{u^2}-\frac{\hat{\epsilon}(\omega)}{c^2}\right)}
			   \left(\bm{u}\omega-\frac{\bm{k}c}{\hat{\epsilon}(\omega)}\right)\frac{d\omega}{u_x}\textcolor{blue}{2\pi}q\int_0^{q_0}dq \xRightarrow{//x}\\
			   %
			  F =& \frac{ie^2}{\pi c^2}\int_{-\infty}^{\infty}\frac{1}{q^2+\omega^2\left(\frac{1}{u^2}-\frac{\hat{\epsilon}(\omega)}{c^2}\right)}  \left(u_x\omega-\frac{k_xc}{\hat{\epsilon}(\omega)}\right) \frac{d\omega}{u_x} \int_0^{q_0} qdq \Rightarrow\\
			   =& \frac{ie^2}{\pi c^2}\int_{-\infty}^{\infty}\frac{1}{q^2+\omega^2\left(\frac{1}{u^2}-\frac{\hat{\epsilon}(\omega)}{c^2}\right)}  
			   \frac{1}{\hat{\epsilon}(\omega)}\left( \hat{\epsilon}(\omega)\omega - \frac{\omega c}{u_x}\right)d\omega\int_0^{q_0}qdq \Rightarrow \\
			   %%
			   %=& \frac{ie^2}{\pi} \int_{-\infty}^{\infty} \int_0^{q_0} \frac{1}{q^2+\omega^2\left(\frac{1}{u^2}-\frac{\hat{\epsilon}(\omega)}{c^2}\right)} 
			   %\frac{1}{\hat{\epsilon}(\omega)} \left(\frac{1}{u_x^2}-\frac{\hat{\epsilon}(\omega)}{c^2}\right)\omega q d\omega dq  \\ 
			   =& \frac{ie^2}{\pi} \int_{-\infty}^{\infty} \int_0^{q_0} 
			   		\frac{1/u_x^2 - \hat{\epsilon}(\omega)/c^2}{\hat{\epsilon}(\omega)\left[q^2+\omega^2\left(\frac{1}{u^2}-\frac{\hat{\epsilon}(\omega)}{c^2}\right)\right]} \omega d\omega qdq \numberthis
	\end{align*}

Η απώλεια ενέργειας ανά μονάδα μήκους της τροχιάς είναι το έργο της δύναμης που ασκείται στο σωματίδιο ανά μονάδα μήκους και	για ένα φορτισμένο σωματίδιο που κινείται με μεγάλη ταχύτητα εντός ενός υλικού δίνεται από την παραπάνω σχέση.
	
	Τώρα θα δούμε υπό ποιές προϋποθέσεις μπορεί να εκπεμφθεί η Cherenkov. 	
Έχουμε ότι ο κυματαριθμός της εκπεμπόμενης ακτινοβολίας θα είναι  $k=\omega \sqrt{\hat{\epsilon}}/c$ καθώς έχουμε υποθέσει διάφανο και ισοτροπικό μέσο και από τα αναπτύγματα Fourier έχουμε ότι αν το σωματίδιό μας κινείται προς την κατεύθυνση x, τότε $\omega= k_xu_x$. Δεδομένου ότι το $k_x$ είναι μία συνιστώσα του k, θα πρέπει να ισχύει 
	\begin{align*}
		k_x <& k \Rightarrow\\ 
		\frac{\omega}{u}<& \frac{\sqrt{\hat{\epsilon(\omega)}}\omega}{c} \Rightarrow\\
		u >& \frac{c}{\sqrt{\hat{\epsilon(\omega)}}} \numberthis
	\end{align*}
	
	Αυτή είναι και η συνθήκη για την εμπομπή της ακτινοβολίας Cherenkov, δηλαδή θα πρέπει η ταχύτητα του σωματιδίου να είναι μεγαλύτερη από την φασική ταχύητα του φωτός στο μέσο.
	
	Επίσης, αν ορίσουμε ως $\theta$ την γωνία μεταξύ των $k_x,\bm{k}$, δηλαδή μεταξύ διεύθυνσης κίνησης του σωματιδίου και εκπομπής ακτινοβολίας, έχουμε ότι 
		\begin{align*}
     		cos\theta =& \frac{k_x}{k } \Rightarrow\\
     				  =& \frac{\omega/u_x}{\omega\sqrt{\hat{\epsilon(\omega)}}/c}\Rightarrow\\
     				  =& \frac{c}{\sqrt{\hat{\epsilon}(\omega)}u_x}	\numberthis
		\end{align*}
		
		Από αυτήν βλέπουμε πως μία δεδομένη τιμή γωνίας θ αντιστοιχεί σε μία τιμή συχνότητας, άρα συνολικά η ακτινοβολία εκπέμπεται σε έναν κώνο με γωνία 2θ.
		Από την παραπάνω σχέση, μπορούμε να δούμε τo εύρος γωνιών $d\theta$ στo οποίο εκπέμπεται ακτινοβολία συχνότητας $d\omega$.
		
		\begin{align*}
			-sin\theta d\theta =& -\frac{c}{2u_x}(\hat{\epsilon}(\omega))^{-3/2} \odv{\hat{\epsilon}}{\omega} d\omega \Rightarrow\\			
			d\theta =& \frac{c}{2u_x\hat{\epsilon}(\omega))^{3/2}}\odv{\hat{\epsilon}}{\omega} d\omega \numberthis
		\end{align*}
		
		Ας επιστρέψουμε τώρα στις απώλειες για να δούμε της εξάρτησή τους από την συχνότητα. Οι απώλειες σε διάστημα συχνότητας $d\omega$ θα είναι από την σχέση (A.20)
		\begin{align*}
			dF =& -d\omega \frac{ie^2}{\pi}\sum_{-\omega,\omega}\omega 
				\left( \frac{1}{c^2}-\frac{1}{\hat{\epsilon}u_x^2}\right)
				\int \frac{qdq}{q^2+\omega^2\left(\frac{1}{u^2}-\frac{\hat{\epsilon}(\omega)}{c^2}\right)}
		\end{align*}
		
		θέτουμε $\xi= q^2+\omega^2\left(\frac{1}{u^2}-\frac{\hat{\epsilon}(\omega)}{c^2}\right)$, άρα γίνεται 
		\begin{align*}
			dF = -d\omega\frac{ie^2}{2\pi}\sum_{-\omega,\omega}\omega \left( \frac{1}{c^2}-\frac{1}{\hat{\epsilon}u_x^2}\right) \int\frac{d\xi}{\xi} \numberthis
		\end{align*}
	
	Για $\xi=0$ περνάμε από έναν πόλο, άρα θα πρέπει να ολοκληρώσουμε πάνω σε μία καμπύλη που περνά γύρω από αυτό. 
	Κάνοντας κάτι τέτοιο είναι σαν να υποθέτουμε πως η διηλεκτρική συνάρτηση $\hat{\epsilon}(\omega)$ έχει και ένα μικρό φανταστικό μέρος, δηλαδή ότι η ακτινοβολία εμφανίζει και μικρή εξασθένιση/απώλεια στο μέσο ανάλογα με το πρόσημό του. Άρα η τιμή του $\xi$ θα πρέπει να μετατοπιστεί λίγο πάνω και λίγο κάτω του πόλου. 
	\begin{align*}
		dF =& -d\omega\frac{ie^2}{2\pi}\left( \frac{1}{c^2}-\frac{1}{\hat{\epsilon}u_x^2}\right) \omega\oint_\gamma \frac{d\xi}{\xi}\Rightarrow\\
		   =&- d\omega\frac{ie^2}{2\pi}\left( \frac{1}{c^2}-\frac{1}{\hat{\epsilon}u_x^2}\right) \omega\left( 2\pi i Res(1/\xi,0) \right)\Rightarrow\\
		 dF  =& \frac{e^2}{ c^2}\left( 1 - \frac{c^2}{\hat{\epsilon u_x^2}}\right) \omega d\omega \numberthis
		   	\end{align*}
		   	
		   	
		   	Τέλος, θα εξετάσουμε την πόλωση της ακτινοβολίας Cherenkov. Από την σχέση (\ref{A.17}) έχουμε ότι $\bm{A} \parallel \bm{u}$. 
		   	Ακόμη $\bm{H}=i\bm{k}\times\bm{A_\bm{k}}$, άρα $\bm{H}\perp\bm{k}$, $\bm{A},\bm{u}$. Συνεπώς, το μαγνητικό πεδίο είναι κάθετο στο επίπεδο που σχηαμτίζουν τα διανύσματα $\bm{k},\bm{u}$, δηλαδή είναι κάθετο στο επίπεδο που σχηματίζει η κατεύθυνση διάδοσης του κύματος και η κατεύθυνση κίνησης του σωματιδίου. Δεδομένου ότι $\bm{E}\perp\bm{H}$, το  $\bm{E}$ θα βρίσκεται στο εν λόγω επίπεδο.