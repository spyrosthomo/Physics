\chapter{Non-autonomous Hamiltonian }
%\addcontentsline{toc}{chapter}{Non-autonomous Hamiltonian }
We will use the theory developed in the above text for systems with n degrees of freedom, in order to study the follwing Hamiltonian with one degree of freedom 
	\begin{equation}\label{eq2.1}
		H(J,\theta,t) = \bar{\omega}J + J^k + V_1cos(n_1\theta-m_1\Omega t) + V_2cos(n_2\theta-m_2\Omega t)
	\end{equation} 
	%
It is obvious that it can be written as the sum of an integrable and a non-integrable Hamiltonian
	\begin{subequations}\label{eq2.2}
			\begin{alignat}{2}
				H_0 =& \bar{\omega}J + J^k\\
				H_1 =& V_1cos(n_1\theta-m_1\Omega t) + V_2cos(n_2\theta-m_2\Omega t)
			\end{alignat}
	\end{subequations}	
	%
This means that if we "turn off" the non-integrable part by setting $V_1=V_2=0$, we can easily find the frequency from the Hamilton's equation of motion for the generalized position
	\begin{align*}\label{eq2.3}
		\omega_0(J) =& \dot{\theta} = \pdv{H_0}{J} \Rightarrow\\
			     =& \bar{\omega} + k J^{k-1} \numberthis
	\end{align*}
	%
The non integrable part can be written as a sum of mean value and an oscillating part 
	\begin{align*}
		H_1 = \langle H_1 \rangle_{\theta,t} + \{ H_1 \}_{\theta,t}
	\end{align*}
But since it is a periodic function with respect to $\theta, t$, we have 
	\begin{align*}
		\langle H_1 \rangle_{\theta,t}
	\end{align*}
and since it is already a sum of cosine functions we will not use the Fourier expansion (\ref{eq18}a) but the following 
	\begin{align}\label{eq2.4}
		\{ H_1 \}_{\theta,t} = H_1 = \sum_{i=1,2} H_{1,i} cos(n_i\theta-m_i\Omega t)
	\end{align}
where 
	\begin{equation}
		H_{1,i} = 
		\begin{cases}
			V_i &, i=1,2\\
			0   &, \text{otherwise}
		\end{cases}
	\end{equation}
Similarly, we write $S_1$ as the following sum
	\begin{align}\label{eq2.6}
		S_1 = \sum_{i} S_{1,i} sin(n_i\theta-m_i\Omega t) 
	\end{align}
If we substitute (\ref{eq2.4}) and (\ref{eq2.6}) into the condition (\ref{eq16}), we have 
	\begin{align*}
		H_1 + \pdv{S_1}{\theta} \omega_0(\bm{\bar{J}})+ \pdv{S_1}{t} = 0 \Rightarrow\\
		\sum_i \left( V_icos(n_i\theta-m_i\Omega t) -m_i\Omega S_{1,i}cos(n_i\theta-m_i\Omega t)  -n_iS_{1,i}cos(n_i\theta-m_i\Omega t)  \cdot \omega_0 \right) = 0 \Rightarrow \\ 
		\sum_i\left( V_i - m_i\Omega S_{1,i}-n_iS_{1,i}\omega_0\right)cos(n_i\theta-m_i\Omega t) =0 \Rightarrow\\ 
	\end{align*}
We want it to hold $\forall i=\{1,2\}$
	\begin{align}\label{eq2.7}
		S_{1,i} = \frac{V_i}{m_i\Omega + n_i(\bar{\omega}+kJ^{k-1})}, i=1,2
	\end{align}
	%
So, for each term of the expansion, we have resonance for the values for which the denominator is zero 
	\begin{align}\label{eq2.8}
		J_{res,i}^{k-1} = - \frac{m_i\Omega + n_i\bar{\omega}}{kn_i}
	\end{align}
%
%
%
%========================================================================
% Q -3 
%========================================================================
%
%
\textcolor{red}{
\line(1,0){450}\\
\begin{center}
	\textbf{QUESTION 3}
\end{center}
\line(1,0){450}\\
}

Also, we can compute the approximate (up to first order) constant of motion $\bar{J}$ from (\ref{eq11}a)
	\begin{align*}
		\bar{J} =& J - \pdv{S_1}{\theta} \Rightarrow\\ 
		        =& J - \sum_{i=1,2}S_{1,i}n_i cos(n_i\theta-m_i\Omega t)\Rightarrow\\ 
		        =& J - \frac{V_1n_1}{m_1\Omega + n_1(\bar{\omega}+kJ^{k-1})}cos(n_1\theta-m_1\Omega t) -\frac{V_2n_2}{m_2\Omega + n_2(\bar{\omega}+kJ^{k-1})} cos(n_2\theta-m_2\Omega t)
	\end{align*}


%===============================================================
%
%
%
\textcolor{red}{
\line(1,0){450}\\
\begin{center}
	\textbf{QUESTION 2}
\end{center}
\line(1,0){450}\\
}
%
%
%========================================================================
% second Q
%========================================================================
%
%
%
If we take either $V_1=0$ or $V_2=0$, then can change the postion variable with the transformation 
	\begin{align}\label{eq2.9}
		\hat{\theta} = n_i\theta-m_i\Omega t, \text{\hspace{.3cm}$i=$1 or 2}
	\end{align}
But if we want to use a Canonical Transformation, we can also choose 
	\begin{align}\label{eq2.10}
		\hat{J} = \alpha J 
	\end{align}
where $\alpha$ is to be determined. In order to demand a Canonical Transformation and by doing so to find $\alpha$, we can use the \textit{symplectic condition} $\tilde{\bm{M}}\bm{J}\bm{M} = \bm{J}$, where $\bm{M}$ is the Jacobian of the transofrmation, or the \textit{Poisson Bracket Condition}, $[\hat{\theta},\hat{J}]=1$, or directly find a generating function. 

	We will use the \textit{Poisson Bracket Condition} in order to determine $\alpha$ and then we will find the generating function of the transformation.
	If we want the transformation to be Canonical, then the following condtion must hold 
	\begin{align*}\label{eq2.11}
		[\hat{\theta},\hat{J}] =& 1 \Rightarrow\\
		\pdv{\hat{J}}{J}\pdv{\hat{\theta}}{\theta} - \cancel{\pdv{\hat{J}}{\theta}}\cancel{\pdv{\hat{\theta}}{J}} =& 1 \xRightarrow{(\ref{eq2.9}),(\ref{eq2.10})}\\ 
		\alpha n_i =& 1 \Rightarrow\\ 
		\alpha =& 1/n_i , \text{\hspace{.6cm}$i=$1 or 2}   \numberthis
	\end{align*}
%
%
In general, the generating function of the Canonical Transformation can be dependent on any pair of new and old variables. We choose it to be $F_2=F_2(\theta,\hat{J})$. Now, the other two variables, can be expressed with $F_2$. For $\hat{\theta}$ 
	\begin{align*}\label{eq2.12}
		\hat{\theta}          =& \pdv{F_2}{\hat{J}} \xRightarrow{(\ref{eq2.9})}\\ 
		n_i\theta-m_i\Omega t =& \pdv{F_2}{\hat{J}}\xRightarrow{\theta - indep.} \\ 		
		F_2 =& (n_i\theta-m_i\Omega t)\hat{J} + f(\theta)\numberthis
	\end{align*}
And for J
	\begin{align*}
		J  =&\pdv{F_2}{\theta} \xRightarrow{(\ref{eq2.12}),(\ref{eq2.10})} \\ 
n_i\hat{J} =& n_i\hat{J}+f'(\theta) \Rightarrow\\ 
		 f(\theta) =& const.				
	\end{align*}
So, we can ignore $f(\theta)$ since in our transformation appears only as a derivative. Finally, the generating function is 
	\begin{equation}\label{eq2.13}
		F_2(\theta,\hat{J}) = (n_i\theta - m_i\Omega t)\hat{J}, \text{\hspace{.3cm}$i=$1 or 2}
	\end{equation}
%
The new Hamiltonian, according to (\ref{eq9}c) is 
	\begin{align*}\label{eq2.14}
		\hat{H}(\hat{\theta},\hat{J}) =& H(\hat{\theta}, \hat{J}) + \partial{F_2}/\partial{t} \Rightarrow\\ 
									  =& (\bar{\omega}n_i-m_i\Omega )\hat{J} + n_i^k\hat{J}^k +V_icos\hat{\theta} \numberthis
	\end{align*}
And since it depends on $\hat{\theta}$, it is not integrable, but again, it can been seen as a sum of an integrable and a non-integrable perturbation. So, from CPT and (\ref{eq11}a), the first order approximation of the new \textit{Action}, is 
	\begin{align*}
		\hat{\bar{J}} = \hat{J} - \pdv{S_1}{\hat{\theta} }
	\end{align*}
Where $S_1$ is the same as the one we calculated (\ref{eq2.6}) for the full Hamiltonian (both $V_1,V_2\neq0$) without the one sine term, thus 
	\begin{align*}
		\hat{\bar{J}} = \hat{J} - \frac{V_i}{m_i\Omega-n_i(\bar{\omega}+k\hat{J}^{k-1})}cos\hat{\theta} , \text{\hspace{.6cm}$i=$1 or 2}   \numberthis
	\end{align*}